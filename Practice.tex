\documentclass[11pt]{article}
\usepackage{amsmath,amssymb,amsthm}
\usepackage{algorithm}
\usepackage[noend]{algpseudocode} 

%---enable russian----

\usepackage[utf8]{inputenc}
\usepackage[russian]{babel}

% PROBABILITY SYMBOLS
\newcommand*\PROB\Pr 
\DeclareMathOperator*{\EXPECT}{\mathbb{E}}


% Sets, Rngs, ets 
\newcommand{\N}{{{\mathbb N}}}
\newcommand{\Z}{{{\mathbb Z}}}
\newcommand{\R}{{{\mathbb R}}}
\newcommand{\Zp}{\ints_p} % Integers modulo p
\newcommand{\Zq}{\ints_q} % Integers modulo q
\newcommand{\Zn}{\ints_N} % Integers modulo N

% Landau 
\newcommand{\bigO}{\mathcal{O}}
\newcommand*{\OLandau}{\bigO}
\newcommand*{\WLandau}{\Omega}
\newcommand*{\xOLandau}{\widetilde{\OLandau}}
\newcommand*{\xWLandau}{\widetilde{\WLandau}}
\newcommand*{\TLandau}{\Theta}
\newcommand*{\xTLandau}{\widetilde{\TLandau}}
\newcommand{\smallo}{o} %technically, an omicron
\newcommand{\softO}{\widetilde{\bigO}}
\newcommand{\wLandau}{\omega}
\newcommand{\negl}{\mathrm{negl}} 

% Misc
\newcommand{\eps}{\varepsilon}
\newcommand{\inprod}[1]{\left\langle #1 \right\rangle}

 
\newcommand{\handout}[5]{
  \noindent
  \begin{center}
  \framebox{
    \vbox{
      \hbox to 5.78in { {\bf Научно-исследовательская практика} \hfill #2 }
      \vspace{4mm}
      \hbox to 5.78in { {\Large \hfill #5  \hfill} }
      \vspace{2mm}
      \hbox to 5.78in { {\em #3 \hfill #4} }
    }
  }
  \end{center}
  \vspace*{4mm}
}

\newcommand{\lecture}[4]{\handout{#1}{#2}{#3}{Scribe: #4}{Быстрый (дихотомический) алгоритм возведения в степень #1}}

\newtheorem{theorem}{Теорема}
\newtheorem{lemma}{Лемма}
\newtheorem{definition}{Определение}
\newtheorem{corollary}{Следствие}
\newtheorem{fact}{Факт}

% 1-inch margins
\topmargin 0pt
\advance \topmargin by -\headheight
\advance \topmargin by -\headsep
\textheight 8.9in
\oddsidemargin 0pt
\evensidemargin \oddsidemargin
\marginparwidth 0.5in
\textwidth 6.5in

\parindent 0in
\parskip 1.5ex

\begin{document}

\lecture{}{Лето 2020}{}{Ганиман Вадим Вадимович}

\section{Теория}

\textbf{ Алгоритмы быстрого возведения в степень (дихотомический алгоритм возведения в степень, бинарный алгоритм возведения в степень)} — алгоритмы, предназначенные для возведения числа x в натуральную степень n за меньшее число умножений, чем это требуется в определении степени. Алгоритмы основаны на том, что для возведения числа x в степень n не обязательно перемножать число x на само себя n раз, а можно перемножать уже вычисленные степени. В частности, если n = 2$^k$ степень двойки, то для возведения в степень n достаточно число возвести в квадрат k раз, затратив при этом k умножений вместо 2$^k$.

\section{Алгоритм}

\begin{algorithm}[ph]
	\caption{Быстрый (дихотомический) алгоритм возведения в степень "Справо-Налево"}
	\label{alg:AlgName}
	\textbf{Ввод:}  X и Y  = (Y$_n$$_-$$_1$  ...  Y$_0$)$_2$  \\
	\textbf{Вывод:} Z = X$^y$
	
	\begin{algorithmic}
		
		\State 1.	 Положить tmp = X 
			\State \ \ \ \ \ \	1.1 Если Y$_0$ = 1 , то положить Z = X 
			\State \ \ \ \ \ \	1.2 Если Y$_0$ = 0 , то положить Z = 1
		\\	2. Для  i от 1 до n-1	
		\\ \ \ \ \ \ \	2.1. Положить tmp = tmp$^2$ .
	  	\\ \ \ \ \ \ \	2.2. Если Y$_1$= 1 , то положить Z = Z * tmp .
		
		\State{3.}
		\Return{Z}
	\end{algorithmic}

\end{algorithm}

В основе лежал данный алгоритм. Для того, чтобы цифры не были столь огромными и их вывод не занимал кучу времени, в своей программе я воспользовался выводом по модулю. Если понадобится, то поменяв только пару строк можно будет сделать программу по оригинальному алгоритму. \\Также не стал приравнивать к Z значения соответствующие пунктам 1.1 и 1.2. Всё это оставил "в руках" \ цикла (где теперь i от 0 до n-1). Даже несмотря на эту поправку, сложность алгоритма не поменялась.

\section{Машина}
Персональный Компьютер, на котором проходили тестирования, следующих основных характеристик:
\\$\triangleright$ Процессор: \textbf{Intel(R) Core(TM) i7-4770 CPU @3.40GHz (4 ядра и 4 потока)}
\\$\triangleright$ Видеоадаптер: \textbf{NVIDIA GeForce GTX 1060 6GB}
\\$\triangleright$ Оперативная память: \textbf{16GB}

\section{Анализ}

\begin{tabular}{|c|c|c|c|c|}
\hline
X & Y & N & Sage Algorithm Time & My Algorithm Time\\
\hline
\hline
172 & 45623654 & 1547445 & 0.01s & 3.16s\\
193 & 65623522 & 5733631 & 0.00s & 4.93s\\
227 & 75623473 & 675631 & 0.00s & 5.95s\\
\hline
\end{tabular}

После прохождения тестирования, появилось впечатление, что сервис, на котором я тестировал sage алгоритм (CoCalc) доставал степени из своей базы данных. Либо основой их сервера лежит Супер-Компьютер. Одно из двух. Так как такие числа не вычисляются за доли секунды. Ну а более крупные цифры не дает протестировать из-за ограничения выделенной памяти CoCalc'ом.


\paragraph{Bibliography}:
\\ $[1]$ Львовский С.М. - Набор и верстка в системе LATEX. 5-е изд. 2014.
\\ $[2]$ Wikipedia: Wikipedia, The Free Encyclopedia [online], http://www.wikipedia.org, June 2020
\\ $[3]$ Zimmermann P., Casamayou A., Cohen N. et al. - Computational Mathematics with SageMath. 2018

\end{document}
